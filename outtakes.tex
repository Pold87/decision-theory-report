The low values of explained variation for x ($R^2$(x; rest))and y
($R^2$(x; rest)) show that the used approach is not promising. This is
not surprising, given the fact that the colors are non-linearly
distributed in the image (see Figure\ref{fig:colordist} for the
analysis of the colors). Two alternatives could be used to improve the
model's performance. Here, we investigate both.
\begin{itemize}
\item Alternative 1: Change the map image such that higher $R^2$ are
  obtained
\item Alternative 2: Use a more powerful model
\end{itemize}


\subsection{Different map image}

\begin{figure}[h]
  \centering
  %\includegraphics[width=0.7\textwidth]{}
  \caption{A different map image that makes use of the used features.}
  \label{fig:newmap}
\end{figure}

In this case, the image was chosen by manual analysis of the
features. A more interesting case would be to generate the image based
on desired properties of the inverse correlation matrix. 
An ideal inverse correlation matrix would look like:

\begin{table}                                   
\centering                                      
\begin{tabular}{l|rrrrr}                    
    red   & 1.30  &       &       &       &      \\  
    green & -0.59 & 2.64  &       &       &      \\
    blue  & 0.02  & -1.81 & 2.37  &       &      \\ 
    x     & 0.14  & -0.02 & 0.01  & 1.02  &      \\ 
    y     & 0.13  & 0.24  & -0.32 & -0.01 & 1.06 \\ 
  \midrule
  $R^2$   & 0.23  & 0.62  & 0.58  & 0.02  & 0.06 \\
  \midrule
          & red   & green & blue  & x     & y
\end{tabular}                                   
\caption{Inverse correlation matrix}                        
\label{table:MyTableLabel}                      
\end{table} 

%$R^2$   & *  & *  & *  & 1.00  & 1.00 \\

One possible solution would be to set all correlations to a high
value.


\subsection{The backward approach}

\begin{figure}[h]
  \centering
  \begin{tikzpicture}[every node={}]
    \node[latent, minimum size=1.2cm] (z1) {$Z_1$} ; %
    \node[latent, right=of z1, minimum size=1.2cm] (z2) {$Z_2$} ; %
    \node[latent, right=of z2, minimum size=1.2cm] (z3) {$Z_3$} ; %
    \node[latent, below right=of z1, minimum size=1.2cm] (x) {$X$} ; %
    \node[latent, below right=of z2, minimum size=1.2cm] (y) {$Y$} ; %
    \edge[-] {z1} {x}; \edge[-] {z1} {y}; \edge[-] {z2} {x}; \edge[-]
    {z2} {y}; \edge[-] {z3} {x}; \edge[-] {z3} {y};
  \end{tikzpicture}
  \caption{The graphical model used for generating ideal images.}
\end{figure}

Knowing the criteria for computing $R^2$, we could construct optimal
images for the given approach. These images will be gradient
images. Since images consist of three channels, we could encode the
information in two channels, and use arbitrary information in the
third channel---and therefore even keep our initial image to a great
extent.
